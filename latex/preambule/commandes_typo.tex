%%%%%%%%%%%%%%%%%%%%%%%%%%%%%%%%%%%%%%%%%%%%%%%%
% Ce fichier contient des commandes pour respecter
% la typographie française, mais également pour 
% les abréviations.
%%%%%%%%%%%%%%%%%%%%%%%%%%%%%%%%%%%%%%%%%%%%%%%%

% Évite veuves et orphelines
\widowpenalty=10000
\clubpenalty=10000 

% Permet de modifier l'espace entre es lignes.
\renewcommand{\baselinestretch}{1.1} 
\onehalfspacing % Permet de mettre le reste du texte avec un interligne de 1,5 (recommandé pour les articles universitaires).

% Chiffres romain
\newcommand{\cRM}[1]{\MakeUppercase{\romannumeral #1}}	% Capital
\newcommand{\cRm}[1]{\textsc{\romannumeral #1}}	% Petit majuscule
\newcommand{\crm}[1]{\romannumeral #1} % Minuscule

% Siècle
\newcommand*{\siecle}[1]{%
	\ifnum #1=1%
		\cRm{#1}\textsuperscript{er}~siècle%
	\else%
		\cRm{#1}\textsuperscript{e}~siècle%
	\fi%
}%
\newcommand*{\siecles}[2]{\cRm{#1}~-~\cRm{#2}\textsuperscript{e}~siècles}

% Guillements
\newcommand{\guill}[1]{\og #1\fg{}}

% Nom d'auteur 
\newcommand\auteur[2]{#1~\textsc{#2}\xspace}

% Abréviations
\newcommand{\ssi}{si et seulement si\xspace}
\newcommand{\cad}{c'est-à-dire\xspace}
\newcommand{\cf}[0]{\textit{cf.}\xspace} % Cf. avec espace à la suite
\newcommand{\apjc}{apr. J.-C.\xspace} % après Jésus-Christ
\newcommand{\avjc}{av. J.-C.\xspace} % avant Jésus-Christ
\newcommand{\Mme}{M\textsuperscript{me}\xspace} % Madame
\newcommand{\Mlle}{M\textsuperscript{lle}\xspace} % Mademoiselle
\newcommand{\numero}{n\textsuperscript{o}\xspace} % numéro
\newcommand{\Num}{N\textsuperscript{o}\xspace} % Numéro
\newcommand{\nums}{n\textsuperscript{os}\xspace} % numéros
\newcommand{\Nums}{N\textsuperscript{os}\xspace} % Numéros
